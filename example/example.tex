\documentclass[
  a4paper,
  fontsize = 11pt,
  english,
  openany,
  twoside = false,
  roman equations,
]{thesis}

\usepackage[english]{babel}

% This should yield a good looking document for b5paper
% \RenewDocumentCommand{\marginlayout}{}{
%   \newgeometry{
%     top=22.9mm,
%     bottom=22.9mm,
%     inner=13.3mm,
%     textwidth=100mm,
%     marginparsep=5mm,
%     marginparwidth=46mm,
%   }%
%   \recalchead%
% }

\usepackage[math]{blindtext}

\addbibresource{references.bib}

% Definition of "notations" glossary style
\newglossary*{notations}{Notations}
\loadglsentries[notations]{notations.tex}

\makeindex[intoc]
\makeglossaries
\glsdisablehyper

\title{Example for the thesis class}
\author{Gaëtan Staquet}

\overfullrule=10pt

\begin{document}
  \frontmatter
  \maketitle

  \tableofcontents

  \mainmatter

  \chapter{Mathematical tools}

  \chapterTOC

  \section{Theorems}

  \begin{definition*}{def:box}
    This is a definition.\index{definition}
  \end{definition*}

  This is some text with a marginnote next to it.\marginnote{Test of a margin note.}

  Likewise with a sidenote.\sidenote{Test of a side note.}

  Test of citations~\sidecite{HopcroftU79,Angluin87}.

  \begin{equation}
    a = b\label{eq:a:b}
  \end{equation}
  \restatableMarginEquation{\aEqualb}{a = b}{eq:a:b}
  \marginnote{\aEqualb}

  \subsection{Restate boxes}

  \restate{def:box}
  \marginnote{\restate{def:box:margin}}

  \section{Restate equations}

  \chapter{Various}

  \section{Floats}

  Floats will always go to the top of a page.
  It is thus impossible to have a figure on the same page a chapter is started.
  If a float is big enough, it will go on a page on its own.
  Otherwise, text will come after it.

  Do not put multiple captions in the same environment.

  \begin{figure}
    \includegraphics[width=.5\totalwidth,keepaspectratio]{example-image-a}
    \caption{A small figure.}
  \end{figure}

  \begin{figure}
    \includegraphics[width=\totalwidth]{example-image-a}
    \caption{A figure.}
  \end{figure}

  \begin{algorithm}
    \caption{An algorithm.}
    \begin{algorithmic}[1]
      \Require Some requirements
      \Ensure Does something
      \Statex
      \While {some condition holds}
        \State Do something
        \Comment{This is a very important step}
      \EndWhile
      \State \Return 4
    \end{algorithmic}
  \end{algorithm}

  \begin{figure}
    \begin{minted}[linenos]{JSON}
{
  "title": "Some JSON document",
  "metadata": {
    "why": "To give an example of minted",
    "lines": 7
  }
}
    \end{minted}
    \caption{A listing (given as a figure).}
  \end{figure}

  \begin{table}
    \begin{tabular}{l r}
      \toprule
      Left & Right
      \\
      \midrule
      Value & Value
      \\
      \bottomrule
    \end{tabular}
    \caption{A \textsf{booktabs} table.}
  \end{table}

  \blindtext[10]
  \marginnote[-5\baselineskip]{Test of a text that hopefully is long enough to require a linebreak}
  \marginnote[\baselineskip]{\begin{tabular}{ll}\toprule VALUE & VALUE\\\bottomrule\end{tabular}}
  \blindtext[10]

  \begin{appendices}
    \Blinddocument
  \end{appendices}

  \backmatter

  \Blinddocument

  \printbibliography

  \printindex\label{index}

  \glsaddall[types={notations}]
  \printglossary[
    type=notations,
    style=long-symbol,
    nonumberlist,
  ]

\end{document}