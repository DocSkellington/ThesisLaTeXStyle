\documentclass[
  % a4paper,
  % openany,
  % twoside = false,
  b5paper,
  fontsize = 11pt,
  english,
  roman equations,
]{thesis}

\usepackage[english]{babel}

\usepackage[math]{blindtext}

\addbibresource{references.bib}

% Definition of "notations" glossary style
\newglossary*{notations}{Notations}
\loadglsentries[notations]{notations.tex}

\makeindex
\makeglossaries
\glsdisablehyper

\title{Example for the thesis class}
\author{Gaëtan Staquet}


\begin{document}
  \frontmatter
  \maketitle

  \tableofcontents

  \mainmatter

  \chapter{Layout and mathematical tools}

  The PDF output of this example is meant to be read alongside the tex file.

  \chapterTOC

  \section{In the margin}

  Use the command \verb!\marginnote! to write a note in the margin.\marginnote{This is an example of a margin note.}
  The \verb!sidenote! command works similarly but also adds a mark.\sidenote{This is a sidenote.}
  It is possible to change the mark.\sidenote[\dagger]{Like this, with a \dagger.}
  Note that it is possible to vertically shift a note and to add a picture.
  \marginnote[-11\baselineskip]{\includegraphics[width=\linewidth,keepaspectratio]{example-image-a}}%
  For instance, the A picture in the margin has been shifted upwards by eleven \verb!\baselineskip!.
  A sidenote can also be vertically shifted but the first argument must be left empty if you want to use the automatically generated number.\sidenote[][-2\baselineskip]{This sidenote has been shifted.}

  The \verb!\sidecite! command to put bibliographic references into the margin works similarly to \verb!\marginnote!, \eg,~\sidecite{Example}.
  References can also be vertically shifted, and like for biblatex commands, a postnote (and a prenote) can be added, \eg,~\sidecite[][Prenote]{Example} and~\sidecite[3\baselineskip][Postnote][Prenote]{Example}.

  \section{Mathematical tools}

  The usual mathematical boxes are defined.

  \begin{theorem*}{thm}
    Theorem statement.\index{theorem}
  \end{theorem*}
  \begin{lemma}{lemma}
    Lemma statement.\index{lemma}
  \end{lemma}
  \begin{proposition}{prop}
    Proposition statement.\index{proposition}
  \end{proposition}
  \begin{restatableMargin}{proposition}{prop}
    Version of \Cref{prop} that goes inside the margin.
  \end{restatableMargin}
  \begin{corollary}{cor}
    Corollary statement.\index{corollary}
  \end{corollary}
  \begin{definition*}{def}
    \ifInMargin{Smaller definition.}{Definition statement.}\index{definition}
  \end{definition*}
  \begin{remark}{rem}
    Remark statement.\index{remark}
  \end{remark}
  \begin{example}{ex}
    Example.\index{example}
  \end{example}
  \begin{proof}
    A proof.
  \end{proof}

  An already defined box can be restated in the body:

  \restate{def}

  Here, \Cref{thm} used the starred variant of the environment.
\marginnote{\restate{thm:margin}}% Note the :margin at the end
  It is thus possible to restate a narrower version of the box inside the margin.
\marginnote{\restate{def:margin}}%
  Similarly, \Cref{def} also used the starred variant but also called the \verb!\ifInMargin! command to change the content of the box depending on whether it is written in the margin or the body.
  While \Cref{prop} did not use the starred variant, an explicit margin version was made via the \verb!restatableMargin! environment.
\marginnote{\restate{prop:margin}}%

  Finally, let us illustrate the fact that we can restate equations inside the margin.
  Let us consider the following \emph{numbered} equation:
  \begin{equation}
    \forall i \in \{1, \dotsc, n\}, \exists j \in \{1, \dotsc, i\} : x_j = x_i + 2.\label{eq}
    \restatableMarginEquation{\eqToRestate}{
      \forall i \in \{1, \dotsc, n\},
      \\
      \exists j \in \{1, \dotsc, i\} :
      \\
      x_j = x_i + 2.
    }{eq}
  \end{equation}
  We call the \verb!restatableMarginEquation! as follows:\sidenote{This code must be within a math environment, typically inside the environment for the original equation.}
  \begin{verbatim}
    \restatableMarginEquation{eqToRestate}{
      \begin{multlined}
        \forall i &\in \{1, \dotsc, n\},
        \\
        \exists j &\in \{1, \dotsc, i\} :
        \\
        x_j &= x_i + 2.
      \end{multlined}
    }{eq}
  \end{verbatim}
  From now on, we can use the macro \verb!\eqToRestate! to restate the equation.
  It can be in the margin
  or in the body (which leads to a ugly result):
\marginnote{\eqToRestate}%
  \eqToRestate

  \section{Floats}

  Floats will always go to the top of a page.
  It is thus impossible to have a figure on the same page a chapter is started.

  Do not put multiple captions in the same environment, or floats will be lost (which triggers an error)!

  \begin{figure}
    \includegraphics[width=.5\totalwidth,keepaspectratio]{example-image-a}
    \caption{A small figure.}
  \end{figure}

  \begin{figure}
    \includegraphics[width=\totalwidth]{example-image-a}
    \caption{A figure.}
  \end{figure}

  \begin{algorithm}
    \caption{An algorithm.}
    \begin{algorithmic}[1]
      \Require Some requirements
      \Ensure Does something
      \Statex
      \While {some condition holds}
        \State Do something
        \Comment{This is a very important step}
      \EndWhile
      \State \Return 4
    \end{algorithmic}
  \end{algorithm}

  \begin{figure}
    \begin{minted}[linenos]{JSON}
{
  "title": "Some JSON document",
  "metadata": {
    "why": "To give an example of minted",
    "lines": 7
  }
}
    \end{minted}
    \caption{A listing (given as a figure).}
  \end{figure}

  \begin{table}
    \begin{tabular}{l r}
      \toprule
      Left & Right
      \\
      \midrule
      Value & Value
      \\
      \bottomrule
    \end{tabular}
    \caption{A \textsf{booktabs} table.}
  \end{table}

  \begin{figure}[p]
    \includegraphics[width=\totalwidth]{example-image-a}
    \caption{A figure on a wide page.}
  \end{figure}

  \Cref{app} simply contains a long text to show how the layout behaves.
  Note that it does not contain anything in the margin.

  \begin{appendices}
    \chapter{Appendix}\label{app}
    \Blindtext[5]
    \Blinditemize
    \Blindtext[5]
    \Blindenumerate
    \Blindtext[5]
    \Blinddescription
    \Blindtext[5]
  \end{appendices}

  \backmatter

  \printbibliography

  \printindex\label{index}

  \glsaddall[types={notations}]
  \printglossary[
    type=notations,
    style=long-symbol,
    nonumberlist,
  ]

\end{document}