% \iffalse meta-comment
%
% Copyright (C) 2024 by Gaëtan Staquet
% -----------------------------------
%
% This file may be distributed and/or modified under the
% conditions of the LaTeX Project Public License, either version 1.3
% of this license or (at your option) any later version.
% The latest version of this license is in:
%
%    http://www.latex-project.org/lppl.txt
%
% and version 1.3 or later is part of all distributions of LaTeX
% version 2005/12/01 or later.
%
% \fi
%
% \iffalse
%<*driver>
\ProvidesFile{\jobname.dtx}
%</driver>
%<class>\NeedsTeXFormat{LaTeX2e}[2022-06-01]
%<class>\ProvidesExplClass{thesis}{2024/07/15}{v1.0}{PhD thesis document class}
%
%<class>\RequirePackage{iftex}
%<class>\RequireLuaTeX
%
%<class>\DeclareOption*{\PassOptionsToClass{\CurrentOption}{scrbook}}
%<class>\ProcessOptions
%<class>\PassOptionsToClass{toc=listof}{scrbook}
%<class>\PassOptionsToClass{toc=index}{scrbook}
%<class>\PassOptionsToClass{toc=bibliography}{scrbook}
%<class>\PassOptionsToClass{parskip=half}{scrbook}
%<class>\PassOptionsToClass{headings=optiontoheadandtoc}{scrbook}
%<class>\LoadClass{scrbook}
%
%<class>\RequirePackage{xparse}
%
%<class>\RequirePackage{thslayout}
%
%<class>\RequirePackage[xindy,original]{imakeidx} % Must be loaded before hyperref
%
%<class>\RequirePackage{xurl}
%<class>\RequirePackage[pdfusetitle]{hyperref}
%<class>\RequirePackage{bookmark}
%<class>\RequirePackage{etoc}

%<class>\RequirePackage{csquotes}
%
%<class>\RequirePackage{amssymb}
%<class>\RequirePackage{amsmath}
%<class>\RequirePackage{amsthm}
%<class>\RequirePackage{thmtools}
%<class>\RequirePackage{mathtools}
%
%<class>\RequirePackage[warnings-off={mathtools-colon,mathtools-overbracket}, math-style=TeX]{unicode-math}
%
%<class>\RequirePackage[nameinlink,capitalize,noabbrev]{cleveref}
%
%<class>\RequirePackage{thsbiblio}
%<class>\RequirePackage{thstheorem}
%
%<class>\RequirePackage[abbreviations, xindy, toc, numberedsection=nameref, postdot, stylemods=longbooktabs]{glossaries-extra}
%
%<*driver>
\documentclass[a4paper]{l3doc}
\usepackage{xparse}
\usepackage{expl3}
\usepackage{etoolbox}
\usepackage{hyperref}
\usepackage[nameinlink]{cleveref}
\begin{document}
  \DisableImplementation
  \DocInput{\jobname.dtx}
\end{document}
%</driver>
% \fi
%
% \GetFileInfo{\jobname.dtx}
%
% \title{The \textsf{thesis} class}
% \author{Gaëtan Staquet}
%
% \maketitle
%
% \begin{documentation}
%
% \begin{abstract}
% This \LaTeX class is meant to be used to typeset a PhD thesis (or a similar
% document) containing various definitions, theorems, and equations.
% Margins are made wider than usual to allow the writer to restate definitions,
% theorems, and equations within them, in order to help the reader.
%
% This class is heavily inspired by the \cls{kaobook} class.
% \end{abstract}
%
% \tableofcontents
%
% \section{Introduction}
%
% This \LaTeX class is meant to be used to typeset a PhD thesis (or a similar
% long document) containing various mathematical environments, such as definitions,
% theorems, and equations.
% In order to ease the reading of the document, margins are made wider than usual
% in order to allow restating those mathematical environments.
% Moreover, instead of footnotes, we recommend using \emph{sidenotes} (that have
% the same behavior but are written in the margin).
%
% A short example can be found on the GitHub repository.\footnote{\url{https://github.com/DocSkellington/ThesisLaTeXStyle}}
% A full example of a document produced with this class is my own PhD
% thesis.\footnote{\url{https://www.gaetanstaquet.com/academic/thesis.pdf}}
%
% For now, the class fully supports A4 and B5 paper formats.
% However, it should be sufficient to re-define the commands \cs{marginlayout}
% and \cs{widelayout} (see \Cref{layout}) to have a functional layout for new
% paper formats.
%
% \section{Class options}
%
% On top of the usual options for \LaTeX classes
% (such as \verb!a4paper!, \verb!b5paper!, \verb!openany!, and \verb!twoside!),
% the class provides a new option:
% \DescribeOption{roman equations}
% \texttt{roman equations}.
% If it is set, equations are numbered using roman numbers.
% This helps to visually differentiate equations from definitions, theorems, \emph{etc}.
%
% We highlight that the class is a derivative of \verb!scrbook!.
% That is, it defaults to a two side document where parts and chapters start on
% an odd page.
%
% \section{Layout}\label{layout}
%
% The main objective of this class is to alter the layout to increase the available
% margin space.
% How much space to reserve is dictated by the following commands.
%
% \begin{function}{\marginlayout, \widelayout}
%   The macro
%   \cs{marginlayout} dictates the layout of pages in \enquote{margin mode}
%   (which is the main mode of the class), while \cs{widelayout} gives the layout
%   for the main matter, part pages, and back matter (bibliography, index, and so on).
%   We refer to the documentation of \pkg{geometry} for the commands defining the
%   layout.
%   In order to have correct header and footer, the command \cs{recalchead} must
%   be called at the end of the macro.
% \end{function}
% The default implementation of both macros (which offer good-looking pages
% for A4 and B5 paper formats) are:
%   \begin{verbatim}
% \NewDocumentCommand{\marginlayout}{}{
%  \ifafourpaper
%    \newgeometry{
%      top=27.4mm,
%      bottom=27.4mm,
%      inner=24.8mm,
%      textwidth=125mm,
%      marginparsep=5.8mm,
%      marginparwidth=46mm,
%    }
%  \fi
%  \ifbfivepaper
%    \newgeometry{
%      top=22.9mm,
%      bottom=22.9mm,
%      inner=13.3mm,
%      textwidth=100mm,
%      marginparsep=5mm,
%      marginparwidth=46mm,
%    }
%  \fi
%  \recalchead
% }
% \NewDocumentCommand{\widelayout}{}{
%  \ifafourpaper
%    \newgeometry{
%      top=27.4mm,
%      bottom=27.4mm,
%      inner=24.8mm,
%      outer=24.8mm,
%      marginparsep=0mm,
%      marginparwidth=0mm,
%    }
%  \fi
%  \ifbfivepaper
%    \newgeometry{
%      top=22.9mm,
%      bottom=22.9mm,
%      inner=13.3mm,
%      outer=13.3mm,
%      marginparsep=0mm,
%      marginparwidth=0mm,
%    }
%  \fi
%  \recalchead
% }
% \end{verbatim}
%
% \begin{function}{\pagelayout}
%   \begin{syntax}
%     \cs{pagelayout} \marg{layout}
%   \end{syntax}
%   Changes the used page layout.
%   The value of \meta{layout} must be either \texttt{margin} or \texttt{wide}.
% \end{function}
%
% \subsection{Writing in the margins}
%
% The class offers two main macros to write inside the margins.
%
% \begin{function}{\marginnote, \sidenote}
%   \begin{syntax}
%     \cs{marginnote} \oarg{shift} \marg{contents}
%     \cs{sidenote} \oarg{mark} \oarg{shift} \marg{contents}
%   \end{syntax}
%   The \cs{marginnote} macro writes \meta{contents} in the margin of the page,
%   while \cs{sidenote} does the same but adds a mark at the position of the macro
%   and the same mark before the contents in the margin (akin to \cs{footnote}).
%
%   If \meta{shift} is absent, the contents will act as a float that aims to
%   vertically align with the macro call.
%   If it is not empty, it must be a length that dictates by how much space
%   the contents must shift \emph{vertically}.
%   It is recommended to use a multiple of \cs{baselineskip}.
%
%   For \cs{sidenote}, if \oarg{mark} is absent or empty, the mark comes from
%   the \texttt{sidenote} counter.
%   If it is not empty, the value is used as the mark.
% \end{function}
%
% \subsection{Floating environments}
%
% The default placement for floating figures, tables, and algorithms is \texttt{t}
% (that is, the top of a page).
% You can override this for a specific environment as usual.
% If you want to change this behavior (or add a default placement for other floats),
% use the \cs{floatplacement} macro (from the \pkg{float} package).
% For instance, \verb!\floatplacement{figure}{b}! changes the default placement
% of figures to the bottom of a page.
%
% \subsection{Algorithms}
%
% The class includes a hack (from \url{https://tex.stackexchange.com/a/147751})
% to add vertical lines inside algorithms typesetted using \pkg{algpseudocode}.
% It is very important to ensure that every line of the algorithm consumes a single
% line in the PDF.
% Otherwise, the vertical lines will go over the spilling text.
%
% \section{Bibliography}
%
% While the usual citation macros (from \pkg{biblatex}) still work, we recommend
% using the following macro to print a short citation in the margin.
%
% \begin{function}{\sidecite, \formatmargincitation}
%   \begin{syntax}
%     \cs{sidecite} \oarg{shift} \oarg{note} \oarg{postnote} \marg{key}
%     \cs{formatmargincitation} \marg{key}
%   \end{syntax}
%   The \cs{sidecite} macro acts as a \pkg{biblatex}-like command that also prints
%   a short version of the citation corresponding to \meta{key} in the margin.
%   The \meta{shift} argument is a length giving the \emph{vertical} shift of
%   the text in the margin (see \cs{marginnote} and \cs{sidenote}).
%   The \meta{note} and \meta{postnote} arguments behave as for \pkg{biblatex}
%   commands: if only \meta{note} is given, it acts as a postnote; if both are
%   given, \meta{note} is a prenote, while \meta{postnote} is the postnote.
%
%   The \cs{formatmargincitation} macro is automatically called by \cs{sidecite}
%   and produces the text to write in the margin.
%   Its default implementation is
%   \begin{verbatim}
%\NewDocumentCommand{\formatmargincitation}{m}
%{\parencite{#1}: \citeauthor*{#1} (\citeyear{#1}), \citetitle{#1}}
%   \end{verbatim}
% \end{function}
%
% The \cls{thesis} class supresses the following fields in the bibliography:
% \texttt{archivePrefix}, \texttt{arxivId}, \texttt{pmid}, \texttt{eprint}, and
% if the type of the reference is a book, \texttt{doi}, via:
% \begin{verbatim}
% \AtEveryBibitem{
%   \clearfield{archivePrefix}
%   \clearfield{arxivId}
%   \clearfield{pmid}
%   \clearfield{eprint}
%   \ifentrytype{book}{\clearfield{doi}}{}
% }
% \end{verbatim}
%
% \section{Mathematical environments}
%
% The class defines environments for theorems, lemmas, propositions, corollaries,
% definitions, assumptions, examples, and remarks.
% We here describe them by focusing on theorems.
%
% \DescribeEnv{theorem}
% The \env{theorem} environment has one optional argument, which is the
% eemph{description} of the theorem (akin to the optional argument for \cls{amsthm}
% environment), and one mandatory argument, which is the label
% (without the call to \cs{label}) of the theorem.
% That is, its syntax is
% \begin{verbatim}
%   \begin{theorem}[Description]{thm:part:chapter:name}
%   Statement
%   \end{theorem}
% \end{verbatim}
% This produces a \pkg{tcolorbox} box, which is also saved (to be able to restate
% the box later), using the label as the name of the saved box for the following
% command.
%
% \begin{function}{\restate}
%   \begin{syntax}
%     \cs{restate} \marg{label}
%   \end{syntax}
%   Restate the box produced by a previous \env{theorem} environment with the
%   same label.
% \end{function}
%
% \DescribeEnv{theorem*}
% The class also provides starred versions of the environments, with the same
% arguments as the unstarred one.
% On top of the already mentioned saved box, another box whose width matches the
% size of the margin is saved, using exactly the same statement.
% The name of that smaller box is \meta{label}\texttt{:margin}.
% See \Cref{misc:conditions} for a macro to provide different texts for the margin
% and the main part of a page.
% \DescribeEnv{restatableMargin}
% Finally, the \env{restatableMargin} can be used
% if the statement has to drastically change for the margin.
% Its syntax is as follows:
% \begin{verbatim}
%   \begin{restatableMargin}{theorem}{label}
%     Statement for margin.
%   \end{restatableMargin}
% \end{verbatim}
% That is, the type of the original box must be provided (here, theorem) and its
% label.
% Again, the name of the produced box that can fit within the margin is
% \meta{label}\texttt{:margin}.
%
% \DescribeEnv{proof}
% Finally, the \env{proof} environment is also provided.
% It is used exactly as the classical \LaTeX environment (that is, without a
% single optional argument).
%
% \subsection{Changing the style}
%
% The class defines the following colors for the boxes:
% \begin{variable}{theoremBackground, theoremFrame}
%   \texttt{theoremBackground} and \texttt{theoremFrame} that give the background
%   and the frame colors for the \env{theorem}, \env{lemma}, \env{proposition},
%   and \env{corollary} environments.
% \end{variable}
% \begin{variable}{definitionBackground, definitionFrame}
%   \texttt{definitionBackground} and \texttt{definitionFrame}
%   that give the background and the frame colors for the
%   \env{definition} and \env{assumption} environments.
% \end{variable}
% \begin{variable}{exampleBackground, exampleFrame}
%   \texttt{exampleBackground} and \texttt{exampleFrame}
%   that give the background and the frame colors for the
%   \env{example} and \env{remark} environments.
% \end{variable}
% Simply redefinying these colors (via \cs{colorlet}, for instance) is enough.
%
% The default styles for the \pkg{tcolorbox} boxes in the body of a page
% are defined as follows:
% \begin{verbatim}
% \tcbset{
%   boxS/.style = {
%     enhanced, breakable = true,
%     left = 2pt, right = 2pt, top = 3pt, bottom = 3pt,
%     before skip balanced = 0.5\baselineskip plus 5pt,
%     after skip balanced = 0.5\baselineskip plus 5pt,
%   },
%   theoremS/.style = {
%     boxS, theorem style = plain,
%     arc = 0pt, boxrule = 0pt, leftrule = 1pt, colframe = theoremFrame,
%     fonttitle = \normalfont\bfseries, description font = \normalfont,
%     coltitle = black, fontupper = \itshape,
%     colback = theoremBackground,
%   },
%   lemmaS/.style = { theoremS, },
%   propositionS/.style = { theoremS, },
%   corollaryS/.style = { theoremS, },
%   definitionS/.style = {
%     theoremS,
%     colframe = definitionFrame, colback = definitionBackground,
%     fontupper = \normalfont,
%   },
%   assumptionS/.style = { definitionS, },
%   exampleS/.style = {
%     theoremS,
%     fonttitle = \normalfont\itshape, fontupper = \normalfont,
%     colframe = exampleFrame, colback = exampleBackground,
%   },
%   remarkS/.style = { exampleS },
% }
% \end{verbatim}
% The default styles for the boxes in the margin are:
% \begin{verbatim}
% \tcbset{
%   theoremMarginS/.style = {
%     theoremS,
%     fonttitle = \normalfont\small\bfseries, fontupper = {\small},
%   },
%   lemmaMarginS/.style = { theoremMarginS, },
%   propositionMarginS/.style = { theoremMarginS, },
%   corollaryMarginS/.style = { theoremMarginS, },
%   definitionMarginS/.style = {
%     theoremMarginS,
%     colframe = definitionFrame, colback = definitionBackground,
%   },
%   assumptionMarginS/.style = { definitionMarginS, },
%   exampleMarginS/.style = {
%     theoremMarginS,
%     fonttitle = \normalfont\small\itshape,
%     colframe = exampleFrame, colback = exampleBackground,
%   },
%   remarkMarginS/.style = { exampleMarginS, },
% }
% \end{verbatim}
% Finally, the style for the \env{proof} environment:
% \begin{verbatim}
% \tcbset{
%   proofS/.style = {
%     boxS,
%     colback=white, arc=0pt, boxrule=0pt, borderline={0.3pt}{-0.25pt}{dotted},
%   }
% }
% \end{verbatim}
% Simply redefine one of those styles, if needed.
%
% \subsection{Restating an equation}
%
% \begin{function}{\restatableMarginEquation}
%   \begin{syntax}
%     \cs{restatableMarginEquation} \marg{name} \marg{equation} \marg{label}
%   \end{syntax}
%   Creates a macro \cs{name} that outputs \meta{equation} (using a linewidth
%   that fits in the margin) and whose tag is given by \meta{label}.
%   When \cs{name} is invoked, there must have been an earlier numbered equation
%   labeled with \meta{label} (otherwise, \LaTeX will produce an error).
% \end{function}
%
% \section{Miscellaneous}
%
% Finally, we give various little tools and customizations made by the class.
%
% \begin{function}{\etal, \etc, \ie, \eg, \cf}
%   The \cs{etal}, \cs{etc}, \cs{ie}, \cs{eg}, and \cs{cf} macros produce
%   italized texts for \emph{et al.}, \emph{etc.}, \emph{i.e.}, \emph{e.g.}, and
%   \emph{cf.}
% \end{function}
%
% \begin{variable}{LinkColor, CiteColor}
%   The colors \texttt{LinkColor} and \texttt{CiteColor} are used to color the
%   text for links (cross-references and URL) and for citations, respectively.
% \end{variable}
%
% \subsection{Chapters}
%
% If a chapter with label \meta{chapter} is written inside a
% \env{appendices} environment, \cs{Cref}\meta{chapter} prints
% \enquote{Appendix~\cs{ref}\meta{chapter}}.
%
% \begin{function}{\chapterTOC}
%   The \cs{chapterTOC} macro outputs the table of contents of the current chapter.
% \end{function}
%
% \subsection{Glossaries and index}
%
% A new style to be used in glossaries produced with the \pkg{glossaries}
% and \pkg{glossaries-extra} packages is provided:
% \texttt{long-symbol}, which can be used to display a (mathematical) symbol
% alongside its description.
% For instance,
% \begin{verbatim}
% \printglossary[
%    type=notations,
%    style=long-symbol,
%    nonumberlist,
%  ]
% \end{verbatim}
%
% \subsection{Conditional macros}\label{misc:conditions}
%
% \begin{function}{\ifafourpaper, \ifbfivepaper}
%   The \cs{ifafourpaper} and \cs{ifbfivepaper} commands can be used to change
%   the contents of the document, depending on the paper format.
%   Do not forget to use \cs{fi} to close the block.
% \end{function}
%
% \begin{function}{\ifInMargin}
%   \begin{syntax}
%     \cs{ifInMargin} \marg{inMargin} \marg{inBody}
%   \end{syntax}
%   If the text is written in the margin, \meta{inMargin} is used; otherwise,
%   \marg{inBody} is.
% \end{function}
%
% \section{Known issues}
%
% Vertical lines in algorithms go over the text, if a line is too long and
% overspills on the next line.
%
% \LaTeX produces overfull warnings when a line is incomplete and finishes by
% a forced linebreak.
% For instance, it is the case for every restated definition/theorem/\emph{etc}.
% in the body of the text (not the margin), or when the line is followed by
% an itemize-like environment.
%
% \end{documentation}
%
% \begin{implementation}
%
\ExplSyntaxOn
%
\newif\ifafourpaper
\newif\ifbfivepaper
\@ifclasswith{scrbook}{a4paper}{\afourpapertrue}{\afourpaperfalse}
\@ifclasswith{scrbook}{b5paper}{\bfivepapertrue}{\bfivepaperfalse}
%
% \begin{macro}{\IfNoValueOrEmptyTF}
%    \begin{macrocode}
\DeclareExpandableDocumentCommand{\IfNoValueOrEmptyTF}{ m m m }{
  \IfNoValueTF{#1}{#2}{\tl_if_empty:nTF {#1} {#2} {#3}}
}
%    \end{macrocode}
% \end{macro}
%
% Print latin words in italics
% \begin{macro}{\hairsp, \etal, \etc, \ie, \eg, \cf}
%    \begin{macrocode}
\NewDocumentCommand{\hairsp}{}{\hspace{1pt}}
\NewDocumentCommand{\etal}{}{\textit{et~al.}\xspace}
\NewDocumentCommand{\etc}{}{\textit{etc.}\xspace}
\NewDocumentCommand{\ie}{}{\textit{i.\nobreak\hairsp{}e.}\xspace}
\NewDocumentCommand{\eg}{}{\textit{e.\nobreak\hairsp{}g.}\xspace}
\NewDocumentCommand{\cf}{}{\textit{cf.}\xspace}
%    \end{macrocode}
% \end{macro}
%
\RequirePackage{appendix}
% To force cleveref to display appendix instead of chapter
\AtBeginEnvironment{appendices}{\crefalias{chapter}{appendix}}
%
% Allow use of \Cref/\cref in section titles
\usepackage[final]{crossreftools}
\pdfstringdefDisableCommands{
  \let\Cref\crtCref
  \let\cref\crtcref
}
%
% Only prints (ref) when using \cref{eq:label}.
\crefname{equation}{}{}
\Crefname{equation}{}{}
%
%------
% Fonts
%------
%
\addtokomafont{part}{\normalfont\scshape\bfseries}
\addtokomafont{partentry}{\normalfont\scshape\bfseries}
\addtokomafont{chapter}{\normalfont\bfseries}
\addtokomafont{chapterentry}{\normalfont\bfseries}
%
\addtokomafont{section}{\normalfont\bfseries}
\addtokomafont{subsection}{\normalfont\bfseries}
\addtokomafont{subsubsection}{\normalfont\bfseries}
\addtokomafont{paragraph}{\normalfont\bfseries}
\setkomafont{descriptionlabel}{\normalfont\bfseries}
%
\RequirePackage[force]{filehook}
\setromanfont[Scale=1.04]{Libertinus Serif}
\setsansfont[Scale=1]{Libertinus Sans}
\setmonofont[Scale=.89]{Liberation Mono}
\setmathfont{Latin Modern Math}
\setmathfont{TeX Gyre Pagella Math}[range=bb]
\setmathfont{STIXTwoMath-Regular}[range=cal]
\AtBeginDocument{% Simplistic redefinitions of missing symbols
  \RenewDocumentCommand{\Coloneq}{}{::=}
  \RenewDocumentCommand{\Coloneqq}{}{::=}
  \setmathfont{STIX}[range=\models]
}
%
% Tighten up space between displays (\eg, equations) and make symmetric (from tufte-latex)
\setlength\abovedisplayskip{6pt plus 2pt minus 4pt}
\setlength\belowdisplayskip{6pt plus 2pt minus 4pt}
\abovedisplayskip 10\p@ \@plus2\p@ \@minus5\p@
\abovedisplayshortskip \z@ \@plus3\p@
\belowdisplayskip \abovedisplayskip
\belowdisplayshortskip 6\p@ \@plus3\p@ \@minus3\p@
%
\ExplSyntaxOff
%
\newglossarystyle{long-symbol}{%
  \setglossarystyle{longheader}%
  \setlength{\glsdescwidth}{.8\hsize}
  \renewcommand*{\glossentry}[2]{%
    \glsentryitem{##1}\glstarget{##1}{\glossentrysymbol{##1}}%
    & \glossentrydesc{##1}.%
    % & ##2%
    \tabularnewline
  }
}
%
\ExplSyntaxOn
%
\colorlet{LinkColor}{DarkOrange!70!black}
\colorlet{CiteColor}{DarkOrange!70!black}
\hypersetup{
  pdfborder={0 0 0}, % Suppress border around pdf
  bookmarksdepth=subsection,
  bookmarksopen=true, % Expand the bookmarks as soon as the pdf file is opened
  linktoc=all, % Toc entries and numbers links to pages
  breaklinks=true,
  colorlinks=true,
  citecolor = CiteColor,
  linkcolor = LinkColor,
  urlcolor = LinkColor,
}
%
% --------
% Table of contents
% --------
%
\renewcommand*{\@pnumwidth}{2em}
%
% \begin{macro}{\chapterTOC}
%    \begin{macrocode}
\NewDocumentCommand{\chapterTOC}{}{
  \begingroup
    \etocarticlestyle
    \setlength{\parskip}{0pt}
    \def\contentsname{Chapter~contents}
    \localtableofcontents
  \endgroup
}
%    \end{macrocode}
% \end{macro}
%
\ExplSyntaxOff
%
% \end{implementation}
%
\endinput