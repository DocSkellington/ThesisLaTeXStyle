% \iffalse meta-comment
%
% Copyright (C) 2024 by Gaëtan Staquet
% -----------------------------------
%
% This file may be distributed and/or modified under the
% conditions of the LaTeX Project Public License, either version 1.3
% of this license or (at your option) any later version.
% The latest version of this license is in:
%
%    http://www.latex-project.org/lppl.txt
%
% and version 1.3 or later is part of all distributions of LaTeX
% version 2005/12/01 or later.
%
% \fi
%
% \iffalse
%<*driver>
\ProvidesFile{\jobname.dtx}
%</driver>
%<package>\NeedsTeXFormat{LaTeX2e}
%<package>\ProvidesExplPackage{thstheorem}{2024/07/15}{v1.0}{PhD thesis: theorem boxes}
%
%<package>\RequirePackage{tikz}
%<package>\RequirePackage[skins, breakable, theorems, magazine]{tcolorbox}

\ExplSyntaxOn

\DeclareKeys[thstheorem]{
  roman~equations.if = @thstheorem@roman,
  roman~equations.usage = preamble
}
\ProcessKeyOptions

\colorlet{theoremBackground}{DarkGray!40!white}
\colorlet{theoremFrame}{black}

\colorlet{definitionBackground}{DarkGray!20!white}
\colorlet{definitionFrame}{DarkGray}

\colorlet{exampleBackground}{DarkGray!10!white}
\colorlet{exampleFrame}{LightGray}

\tcbset{
  boxS/.style = {
    enhanced,
    breakable = true,
    % Padding
    left = 2pt,
    right = 2pt,
    top = 3pt,
    bottom = 3pt,
    % Space before and after
    before~skip~balanced = 0.5\baselineskip~plus~5pt,
    after~skip~balanced = 0.5\baselineskip~plus~5pt,
  },
  % For main body
  theoremS/.style = {
    boxS,
    theorem~style = plain, % No break after theorem title
    % Border
    arc = 0pt,
    boxrule = 0pt,
    leftrule = 1pt,
    colframe = theoremFrame,
    % Title
    fonttitle = \normalfont\bfseries,
    description~font = \normalfont,
    coltitle = black,
    % Main text
    fontupper = \itshape,
    colback = theoremBackground,
  },
  lemmaS/.style = {
    theoremS,
  },
  propositionS/.style = {
    theoremS,
  },
  corollaryS/.style = {
    theoremS,
  },
  definitionS/.style = {
    theoremS,
    colframe = definitionFrame,
    colback = definitionBackground,
    fontupper = \normalfont,
  },
  assumptionS/.style = {
    definitionS,
  },
  exampleS/.style = {
    theoremS,
    fonttitle = \normalfont\itshape,
    fontupper = \normalfont,
    colframe = exampleFrame,
    colback = exampleBackground,
  },
  remarkS/.style = {
    exampleS
  },
  proofS/.style = {
    boxS,
    colback=white,
    % Border
    arc=0pt,
    boxrule=0pt,
    borderline={0.3pt}{-0.25pt}{dotted},
  },
  % For margin
  theoremMarginS/.style = {
    theoremS,
    fonttitle = \normalfont\small\bfseries,
    fontupper = {\small},
  },
  lemmaMarginS/.style = {
    theoremMarginS,
  },
  propositionMarginS/.style = {
    theoremMarginS,
  },
  corollaryMarginS/.style = {
    theoremMarginS,
  },
  definitionMarginS/.style = {
    theoremMarginS,
    colframe = definitionFrame,
    colback = definitionBackground,
  },
  assumptionMarginS/.style = {
    definitionMarginS,
  },
  exampleMarginS/.style = {
    theoremMarginS,
    fonttitle = \normalfont\small\itshape,
    colframe = exampleFrame,
    colback = exampleBackground,
  },
  remarkMarginS/.style = {
    exampleMarginS
  },
}

\theoremstyle{plain}
\newtheorem{theoremInner}{Theorem}[section]
\newtheorem{lemmaInner}[theoremInner]{Lemma}
\newtheorem{propositionInner}[theoremInner]{Proposition}
\newtheorem{corollaryInner}[theoremInner]{Corollary}

\theoremstyle{definition}
\newtheorem{definitionInner}[theoremInner]{Definition}
\newtheorem{assumptionInner}[theoremInner]{Assumption}

\theoremstyle{remark}
\newtheorem{exampleInner}[theoremInner]{Example}
\newtheorem{remarkInner}[theoremInner]{Remark}

\crefname{theoremInner}{theorem}{theorems}
\crefname{lemmaInner}{lemma}{lemmas}
\crefname{propositionInner}{proposition}{propositions}
\crefname{corollaryInner}{corollary}{corollaries}
\crefname{definitionInner}{definition}{definitions}
\crefname{assumption}{assumption}{assumptions}
\crefname{assumptionInner}{assumption}{assumptions}
\crefname{exampleInner}{example}{examples}
\crefname{remarkInner}{remark}{remarks}

\tcolorboxenvironment{theoremInner}{theoremS}
\tcolorboxenvironment{lemmaInner}{lemmaS}
\tcolorboxenvironment{propositionInner}{propositionS}
\tcolorboxenvironment{corollaryInner}{corollaryS}
\tcolorboxenvironment{definitionInner}{definitionS}
\tcolorboxenvironment{assumptionInner}{assumptionS}
\tcolorboxenvironment{exampleInner}{exampleS}
\tcolorboxenvironment{remarkInner}{remarkS}
\tcolorboxenvironment{proof}{proofS}

% {type} {label of original}
% Description of the environment is not retrieved (e.g., Definition 3.2.1 [Description of the definition])
\NewDocumentEnvironment{ths@restatable}{ m m }{
  \newboxarray{#2}
  \begin{tcolorbox}[
    reset~and~store~to~box~array = {#2},
    width = \headtextwidth,
    #1S,
    title = {\Cref*{#2}.},
  ]
}{
  \end{tcolorbox}
}
\NewDocumentEnvironment{ths@restatable*}{ m m }{
  \newboxarray{#2:margin}
  \begin{tcolorbox}[
    reset~and~store~to~box~array = {#2:margin},
    width = \marginparwidth,
    #1MarginS,
    title = {\Cref*{#2}.},
  ]
}{
  \end{tcolorbox}
}

% {type} {label} {description} {margin?} {body}
\NewDocumentCommand{\ths@thm}{ m m m m +m }{
  \IfValueTF{#3}{
    \begin{#1Inner}[#3]\label{#2}
      #5
    \end{#1Inner}
  }{
    \begin{#1Inner}\label{#2}
      #5
    \end{#1Inner}
  }
  \begin{ths@restatable}{#1}{#2}
    #5
  \end{ths@restatable}
  \IfBooleanT{#4}{
    \setlist[1]{leftmargin=*}
    \begin{ths@restatable*}{#1}{#2}
      #5
    \end{ths@restatable*}
  }
}

% All environments require the label!
\NewDocumentEnvironment{theorem}{ o m +b }{\ths@thm{theorem}{#2}{#1}{\BooleanFalse}{#3}}{}
\NewDocumentEnvironment{theorem*}{ o m +b }{\ths@thm{theorem}{#2}{#1}{\BooleanTrue}{#3}}{}
\NewDocumentEnvironment{lemma}{ o m +b }{\ths@thm{lemma}{#2}{#1}{\BooleanFalse}{#3}}{}
\NewDocumentEnvironment{lemma*}{ o m +b }{\ths@thm{lemma}{#2}{#1}{\BooleanTrue}{#3}}{}
\NewDocumentEnvironment{proposition}{ o m +b }{\ths@thm{proposition}{#2}{#1}{\BooleanFalse}{#3}}{}
\NewDocumentEnvironment{proposition*}{ o m +b }{\ths@thm{proposition}{#2}{#1}{\BooleanTrue}{#3}}{}
\NewDocumentEnvironment{corollary}{ o m +b }{\ths@thm{corollary}{#2}{#1}{\BooleanFalse}{#3}}{}
\NewDocumentEnvironment{corollary*}{ o m +b }{\ths@thm{corollary}{#2}{#1}{\BooleanTrue}{#3}}{}
\NewDocumentEnvironment{definition}{ o m +b }{\ths@thm{definition}{#2}{#1}{\BooleanFalse}{#3}}{}
\NewDocumentEnvironment{definition*}{ o m +b }{\ths@thm{definition}{#2}{#1}{\BooleanTrue}{#3}}{}
\NewDocumentEnvironment{assumption}{ o m +b }{\ths@thm{assumption}{#2}{#1}{\BooleanFalse}{#3}}{}
\NewDocumentEnvironment{assumption*}{ o m +b }{\ths@thm{assumption}{#2}{#1}{\BooleanTrue}{#3}}{}
\NewDocumentEnvironment{remark}{ o m +b }{\ths@thm{remark}{#2}{#1}{\BooleanFalse}{#3}}{}
\NewDocumentEnvironment{remark*}{ o m +b }{\ths@thm{remark}{#2}{#1}{\BooleanTrue}{#3}}{}
\NewDocumentEnvironment{example}{ o m +b }{\ths@thm{example}{#2}{#1}{\BooleanFalse}{#3}}{}
\NewDocumentEnvironment{example*}{ o m +b }{\ths@thm{example}{#2}{#1}{\BooleanTrue}{#3}}{}

\renewcommand{\qedsymbol}{$\square$} % I don't know why but the square is removed by something

% {type} {label of original}
\NewDocumentEnvironment{restatableMargin}{ m m }{
  % \setlength{\labelindent}{0em}
  \setlist[1]{leftmargin=*}
  \begin{ths@restatable*}{#1}{#2}
}{
  \end{ths@restatable*}
}

\NewDocumentCommand{\restate}{m}{\useboxarray[#1]{1}}

% Changing the way equations are labeled to use roman numbers
\if@thstheorem@roman
  \renewcommand{\theequation}{\thesection.\roman{equation}}
  \@addtoreset{equation}{section}
\else
  \counterwithin{equation}{section}
\fi


% Restatable equations.
% It creates a macro (that does not take any arguments) that outputs the same equation with the same numbering
\NewDocumentCommand{\restatableMarginEquation}{ m m m }{
  \cs_gset:Npn #1 {
    \footnotesize
    \@fleqntrue
    \begin{equation*}
      \begin{split}
        #2
      \end{split}
      \tag{\ref{#3}}
    \end{equation*}
  }
}
\ExplSyntaxOff

\endinput